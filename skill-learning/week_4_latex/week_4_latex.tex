\documentclass{article}
%ctexart, ctexrep, ctexbook等文档类自动支持中文。
%基础文档类有三种:article, report, book
%大括号里面也可以填字号,纸张类型,是否双面等。

\usepackage{amsmath}
\usepackage{graphicx}
\usepackage{ctex}

\usepackage[left=0.5in,right=0.5in,top=0.5in,bottom=0.5in]{geometry}
%使用margin=1in也能达到同样的效果。
%使用hmargin=1in,vmargin=1in也能达到同样效果
%在装订双面文档(例如book类),inner是靠近书脊的边,在单页模式下等于left,双页模式下得看单双页。可以用inner和outer参数进行定义

%先用usepackage引入包,然后再\geometry命令填入也可。此时[]变成{}

\usepackage{multicol}

%更多的宏包
%https://zhuanlan.zhihu.com/p/43981639

\title{week 4: latex}
\author{Mayrain}
\date{\today}

\begin{document}
\maketitle
\section{What is latex}
\noindent
latex的前身是tex,是一种排版软件,用于生成高质量的文档,比如科技论文,书籍等。latex是tex的一种宏包,是一种对tex的封装,使得tex更加容易使用。\\
latex的优点是可以生成高质量的文档,而且可以使用代码的方式来排版文档,可以很方便的生成数学公式,表格,图片等。\\
latex的缺点是学习曲线比较陡峭,而且不适合用来写小文档,比如笔记等。我用tex做笔记的原因主要是为了熟练掌握其技巧。\\
以上都是copilot写的,我只是复制粘贴了一下(笑)。\\
\\
一般来说发行版就是打包好的套装,包含了latex引擎,宏包,字体等。\\
也可以直接安装引擎,比如xetex,pdftex。他们起到编译器的作用。编写的方法是随意的,任何一个文本编辑器都可以完成这个事情。\\
\section{latex command}
\noindent
latex的命令以\verb|\|开头。而且对大小写敏感\\
这里的\verb|\|最好是用verb命令,而不用text或者\verb|\textbackslash|。中者无法实现(命令在大括号中也有效),后者太麻烦了。\\
latex可以用\{\}限定作用范围,而且也可以用[]表达可选参数。在\{\}中的表达是必选的。\\
环境就是一种命令!\\
\section{latex space}
\noindent
有关latex中的空格/段落,需要注意:\\
\begin{itemize}
    \item 1个或多个空格,latex会当做一个空格处理
    \item 段首空格不处理。我们必须要用\verb|\hspace{2em}|这样的命令才能使其空格。
    \item 换行符视为一个空格,也就是说只换行,latex不会换行,必须要再换一行以空出一行。(连续两个换行符,latex会认为是一个段落的结束,会自动空出一行。)
    \item \verb|\par|或者是空行,latex会认为是一个段落的结束,会自动空出一行。
    \item \verb|\\ \newline|则是所谓断行,相当于段落内换行,不产生新段落。
    \item \verb|\newpage \clearpage|则是手动断页,但是前者在双栏状态下左页换右页,还在同一页;后者则是直接该页都不要了。
\end{itemize}
\section{latex strange symbol}
\subsection{quotation symbol}
\noindent
latex的重要特点是“非二义性”。在word中,左右引号是自动识别的,但是对于latex中这样的识别仍然不够准确。如果直接键入单/双引号,latex只会将其识别为右引号,而没有左引号。\\
左引号的出现应当使用反引号,也就是\verb|`|,输入上:``We bands of brothers" 和`We bands of brothers'才是正确的。\\
不过,双引号也可以使用两个单引号来作为右引号,例如``We bands of brothers''。\\
如果搞错了,情况就是这样:\\
"We bands of brothers"\\
'We bands of brothers'\\
危险的地方在于这甚至不会报错。
\subsection{prevent connecting(防止连字)}
\noindent
在两个字母中间使用\{\}可以避免连字,例如:\\
difficult\\
dif{}f{}icult\\
下面的单词在两个f,f和i之间加入了\{\}用于规避连字,看起来美观多了。
\subsection{mandatory distance}
\noindent
长度单位有pt, in, cm, mm, em(当前M宽度), ex(当前x高度)\\
几种修改:\\
1. 行距:\verb|\linespread{factor}|。默认行距为1.2倍字体大小。\\
2. 水平间距:\verb|\hspace{length}|。\\
\hspace*{2em}\verb|\hspace*{length}|可以规避当只有一边有内容时,距离会被自动吞掉的情况。\\
\hspace*{2em}\verb|\quad|代表1em,两个q则是2em\\
\hspace*{2em}\verb|\hspace{\fill}lol|这样的操作则会将lol放在本行最后,中间全部用空格填充。\\
3. 竖直间距:\verb|\vspace|\\
\hspace*{2em}也可以使用类似于\verb|\\[1em]|这样的命令来换行时插入垂直间距。\\
4. 分栏:\\
全局分栏:\verb|\twocolumn \onecolumn|很明显了\\
局部分栏:需要引入宏包multicol,然后使用begin包含multicols环境,注释里可以看到。

%\begin{multicols}{2}
%   [
%       \section{First Section}
%       All human things are subject to decay. And when fate summons, Monarchs must obey.
%   ]
%   \blindtext\blindtext
%\end{multicols}
%\end{document}

%相关链接:https://blog.csdn.net/xovee/article/details/123087349
\section{docu structure}
\subsection{Chap. and index}
\end{document}