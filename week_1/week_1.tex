\documentclass{article}

\usepackage{amsmath}
\usepackage{ctex}
\usepackage{graphicx}

\begin{document}
第一周主要讲解terminal和shell的使用。如何用命令行控制电脑。命令行应当与计算机共存,所以必然要学。\\
我记得老师说到过nvim,也可以用lazyvim(是叫这个名字吗?)

对于markdown来说,它是html的一个子集,所以可以用html的标签来写markdown。当然也是html的化简。\\

另外我们也需要GNU Maker。这东西是用来自动编译的,似乎类似于makefile?这一点我可能不会在辅学去学而选择在b站大学自己自学。

几个注意事项:
1. 电脑信息?
2. 权限?sudo?
3. 环境变量的处理?
- shell查找指令->找到对应二进制文件->执行程序
环境变量的作用在第二部分,它需要去查找文件。所以为什么一些编译会失败,是因为没有将编译器加入到环境变量中,导致环境变量找不到编译器进而无法执行编译指令。
4. 分盘的作用在于将系统文件分配到执行较快的一部分以提高反应效率。
5. 以课程为单位建立文件夹。
6. 回避中文名。
\end{document}