\documentclass{article}

\usepackage{ctex}
\usepackage{amsmath}
\usepackage{graphicx}

\begin{document}
\noindent
本次学习的内容:\\
shell,terminal,vim,GNU make\\

\noindent
注册表:电脑中的巨大数据库,所有的设置都在注册表中有值。32年之后,仍然需要更改注册表。\\
windows也没有包管理器。windows是闭源软件。\\
软件上兼容性还行,但是硬件不行。仅微软在开发。CPU架构,arm64,x86,x86-64云云。\\
windows不能使用nul作为文件名。至今为止仍然如此。\\
\begin{align}
    linux & ->android \notag   \\
          & ->arduino   \notag
\end{align}

\noindent
有关terminal和shell
terminal,是一个模拟传统终端的东西。他本质上还是个应用程序,提供了窗口和交互功能的应用程序。\\
只是terminal他内部运行的是shell而已,shell才是执行命令的。\\
cmd不算terminal,而算是shell。windows terminal才是。很多系统都会自带terminal的

\noindent
shell:壳。一个应用程序。它本身也是用户与系统内核交互的界面而已。
接受+解析输入,交给操作系统执行,并返回输出。
比较好用的shell:zsh(z shell)
pwsh 7(问题来了,powershell和pwsh有什么区别?)
\end{document}