\documentclass{article}

\usepackage{ctex}
\usepackage{amsmath}
\usepackage{graphicx}

\begin{document}
\noindent
本次学习的内容:\\
shell,terminal,vim,GNU make\\

\noindent
注册表:电脑中的巨大数据库,所有的设置都在注册表中有值。32年之后,仍然需要更改注册表。\\
windows也没有包管理器。windows是闭源软件。\\
软件上兼容性还行,但是硬件不行。仅微软在开发。CPU架构,arm64,x86,x86-64云云。\\
windows不能使用nul作为文件名。至今为止仍然如此。\\
\begin{align}
    linux & ->android \notag   \\
          & ->arduino   \notag
\end{align}

\noindent
有关terminal和shell
terminal,是一个模拟传统终端的东西。他本质上还是个应用程序,提供了窗口和交互功能的应用程序。\\
只是terminal他内部运行的是shell而已,shell才是执行命令的。\\
cmd不算terminal,而算是shell。windows terminal才是。很多系统都会自带terminal的

\noindent
shell:壳。一个应用程序。它本身也是用户与系统内核交互的界面而已。
接受+解析输入,交给操作系统执行,并返回输出。
比较好用的shell:zsh(z shell)
pwsh 7(问题来了,powershell和pwsh有什么区别?)

\noindent
对于命令行来说,也可以用命令行中查看网页。他其实就是一种操作计算机的方式。对于ssh里链接服务器,那更是如此。\\
\begin{center}
    dev cpp -> vscode \& gcc\\
\end{center}
这种转换可以让你熟知编译原理。另外你最好需要知道json,否则每次都配就很恶心。\\

\noindent
插件部分:\\
oh my zsh, powerlevel10k(p10k), git, sudo, z, zsh-autosuggestions, zsh-syntax-highlighting\\
可参看tony crane的博客。\\

\noindent
shell的用法:\\
命令与位置非常有关系。rm -rf ./ 这个命令让你可以删除当前目录下的所有文件。~则代表home。\\
另外,*nix系的路径分隔符是/,而windows是\textbackslash。\\
*nix系没有分盘,只有一个根目录。,但是windows则有盘的概念\\

\noindent
prompt:命令提示符,就是跳出来的那一行,告诉我们信息(权限和所在目录)并等待我们输入。\\

\noindent
相应的一些命令:\\
pwd:获取当前路径。\\
cd:切换路径 ../代表上一级目录,./代表当前目录,/代表根目录,~代表home目录。\\
以根目录(/)为起始的路径叫做绝对路径,而以当前目录为起始的路径叫做相对路径。\\

\noindent
ls:列出当前目录下的文件。\\
\qquad -a:列出所有文件,包括隐藏文件。\\
\qquad -l:列出详细信息。包含权限,大小,时间。\\
\qquad 我们也可以用-al和-a -l来做。\\

\noindent
mkdir:创建目录。\\
touch filename:创建文件。这里可以跟上多个文件名。\\

\noindent
cp src dest:复制文件。不加参数是不能复制文件夹的。\\
这里的dest两种情况:\\
1. dest为目录,则复制到该目录下。\\
2. dest为文件,则复制并重命名,并归入当前目录。本质上和1是一样的,这里用到的是相对路径。\\
\qquad -r:递归复制。\\

\noindent
mv src dest:移动文件。它也可以被用于重命名\\
\qquad -r:递归移动。\\

\noindent
rm filename:删除文件。\\
\qquad -r:递归删除。\\
\qquad -f:强制删除。\\
\qquad -rf:强制递归删除。这命令非常危险。\\

\noindent
find path -name filename:在path下查找文件名为filename的文件。\\
\qquad 模糊查找*也可以在此处使用。但是要用引号(单双都可以)括起来。\\

\noindent
cat filename:查看文件内容。将会在终端中全部打出。\\
\qquad *concatenate的缩写。\\
\qquad 也可以用于拼接文件。但不是真的拼在了一起,只是一起输出了而已。\\
\qquad -n:显示行号。\\

\noindent
head filename:查看文件头部。默认显示前10行。\\
\qquad -n:显示前n行。\\

\noindent
tail filename:查看文件尾部。默认显示后10行。\\
\qquad -n:显示后n行。\\

\noindent
less filename:分页显示文件内容。\\
\qquad 可以实现查找和更好的翻页,还可以使用滚轮。\\

\noindent
more filename:分页显示文件内容。\\
\qquad 这个就just so so了。\\

\noindent
hexdump filename:以16进制显示文件内容。\\
\qquad -C:以16进制显示文件内容,并且显示ASCII码。\\
\qquad -n N:只显示前N个字节。\\

\noindent
其他文档:\\
man xxx:查看xxx的帮助文档。他的功能强大到C语言的函数都能看。\\
echo xxx:输出xxx。\\

\noindent
clear:清屏。\\
whoami:查看当前用户。\\

\noindent

\end{document}