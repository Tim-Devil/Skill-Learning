\documentclass{article}

\usepackage{ctex}
\usepackage{graphicx}
\usepackage{amsmath}

\title{Week 3: git}
\author{Mayrain}
\date{\today}

\begin{document}

\maketitle

\section{Git}

\subsection{What is git}

\noindent
一个分布式的版本控制系统。分布式代表不需要联网,版本控制代表可以回溯文件修改历史。\\
有趣的部分:git是自托管的,也就是说git他自己的代码就是放在git仓库里的。现在你甚至可以在github上看到git的源代码。实现是用1000多行代码完成的。\\
git自带的git bash是一个命令行工具,可以用来操作git。他也很有用。\\

\subsection{How git works}

\noindent
\begin{align}
    working\quad directory & -> staging\quad area & ->  git  repository \notag \\
                           & add                  & commit \notag
\end{align}
他妈的这一段真丑啊我去,我得搞明白为什么排版会变成这样,但是不是现在。\\
push: 本地仓库 -> 远程仓库\\
pull: 远程仓库 -> 本地仓库\\
在这里课程讲的很简单,最好参见网络教程。\\
\\
利用git init folder可以新建一个文件夹并将其转化为git仓库。\\
git文件有三种状态:
\begin{itemize}
    \item untracked: 未跟踪
    \item modified: 已修改
    \item staged: 已暂存
    \item ignored: 已忽略
\end{itemize}
这里的ignored是指git不会跟踪这个文件,也就是说这个文件不会出现在git status的结果中。他被存储在gitignore文件。一般我们都在这里加一些规则。\\
github/gitignore: 这里有很多gitignore的规则,可以直接复制。\\
\\
commit message standards:\\
angular/angular:CONTRIBUTING.md\\
\\
作者的话:\\
老天,连git的commit的message都在github上有标准化的规定,不得不说这就是程序员的思维:机械而规范。让我想起了github上有名的“how to ask questions wisely”系列。很有意思。\\

\end{document}